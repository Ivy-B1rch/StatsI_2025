of\documentclass[12pt,letterpaper]{article}
\usepackage{graphicx,textcomp}
\usepackage{natbib}
\usepackage{setspace}
\usepackage{fullpage}
\usepackage{color}
\usepackage[reqno]{amsmath}
\usepackage{amsthm}
\usepackage{fancyvrb}
\usepackage{amssymb,enumerate}
\usepackage[all]{xy}
\usepackage{endnotes}
\usepackage{lscape}
\newtheorem{com}{Comment}
\usepackage{float}
\usepackage{hyperref}
\newtheorem{lem} {Lemma}
\newtheorem{prop}{Proposition}
\newtheorem{thm}{Theorem}
\newtheorem{defn}{Definition}
\newtheorem{cor}{Corollary}
\newtheorem{obs}{Observation}
\usepackage[compact]{titlesec}
\usepackage{dcolumn}
\usepackage{tikz}
\usetikzlibrary{arrows}
\usepackage{multirow}
\usepackage{xcolor}
\newcolumntype{.}{D{.}{.}{-1}}
\newcolumntype{d}[1]{D{.}{.}{#1}}
\definecolor{light-gray}{gray}{0.65}
\usepackage{url}
\usepackage{listings}
\usepackage{color}

\definecolor{codegreen}{rgb}{0,0.6,0}
\definecolor{codegray}{rgb}{0.5,0.5,0.5}
\definecolor{codepurple}{rgb}{0.58,0,0.82}
\definecolor{backcolour}{rgb}{0.95,0.95,0.92}

\lstdefinestyle{mystyle}{
	backgroundcolor=\color{backcolour},   
	commentstyle=\color{codegreen},
	keywordstyle=\color{magenta},
	numberstyle=\tiny\color{codegray},
	stringstyle=\color{codepurple},
	basicstyle=\footnotesize,
	breakatwhitespace=false,         
	breaklines=true,                 
	captionpos=b,                    
	keepspaces=true,                 
	numbers=left,                    
	numbersep=5pt,                  
	showspaces=false,                
	showstringspaces=false,
	showtabs=false,                  
	tabsize=2
}
\lstset{style=mystyle}
\newcommand{\Sref}[1]{Section~\ref{#1}}
\newtheorem{hyp}{Hypothesis}

\title{Problem Set 3}
\date{Due: November 13, 2025}
\author{Applied Stats/Quant Methods 1}


\begin{document}
	\maketitle
	\section*{Instructions}
	\begin{itemize}
		\item Please show your work! You may lose points by simply writing in the answer. If the problem requires you to execute commands in \texttt{R}, please include the code you used to get your answers. Please also include the \texttt{.R} file that contains your code. If you are not sure if work needs to be shown for a particular problem, please ask.
	\item Your homework should be submitted electronically on GitHub.
	\item This problem set is due before 23:59 on Thursday November 13, 2025. No late assignments will be accepted.

	\end{itemize}

		\vspace{.25cm}
	
\noindent In this problem set, you will run several regressions and create an add variable plot (see the lecture slides) in \texttt{R} using the \texttt{incumbents\_subset.csv} dataset. Include all of your code.

	\vspace{.5cm}
\section*{Question 1}
\vspace{.25cm}
\noindent We are interested in knowing how the difference in campaign spending between incumbent and challenger affects the incumbent's vote share. 
	\begin{enumerate}
		\item Run a regression where the outcome variable is \texttt{voteshare} and the explanatory variable is \texttt{difflog}.	\vspace{5cm}
		inc.sub <- read.csv("https://raw.githubusercontent.com/ASDS-TCD/StatsI_2025/main/datasets/incumbents_subset.csv")
		

		VoteDifReg <- lm(voteshare~difflog, inc.sub)
		summary(VoteDifReg)
		Coefficients:
							Estimate 	Std. Error t value Pr(>|t|)    
		(Intercept) 0.579031   0.002251  257.19   <2e-16 ***
		difflog     0.041666   0.000968   43.04   <2e-16 ***
		
		
		Using lm to run the regression of the two variables.
		
		\item Make a scatterplot of the two variables and add the regression line. 	\vspace{7cm}
		
		ggplot(inc.sub, aes(x= difflog, y = voteshare)) + 
			geom_point() + 
			geom_smooth(method='lm')
		\includegraphics[width=1.1\textwidth]{../../../../my_answers/PS3-1.png}
		
		
		\item Save the residuals of the model in a separate object.	\vspace{7cm}
		
		VoteDifResid <- VoteDifReg\$residuals
		
		
		\item Save the residuals of the model in a separate object.	\vspace{7cm}
		
		VoteDifResid <- VoteDifReg\$residuals
		\item Write the prediction equation.
		
		y = Beta0 + Beta1*x
		
		voteshare = 0.57903 + 0.04167*difflog
		
	\end{enumerate}
	
\newpage

\section*{Question 2}
\noindent We are interested in knowing how the difference between incumbent and challenger's spending and the vote share of the presidential candidate of the incumbent's party are related.	\vspace{.25cm}
	\begin{enumerate}
		\item Run a regression where the outcome variable is \texttt{presvote} and the explanatory variable is \texttt{difflog}.	\vspace{5cm}
		PresVoteReg <-lm(presvote~difflog, inc.sub)
		summary()PresVoteReg)
		Coefficients:
							Estimate 	Std. Error t value 	Pr(>|t|)    
		(Intercept) 0.507583   0.003161  160.60   <2e-16 ***
		difflog     0.023837   0.001359   17.54   <2e-16 ***
		
		Using lm to run the regression of the two variables.
		
		\item Make a scatterplot of the two variables and add the regression line. 	\vspace{5cm}
		ggplot(inc.sub, aes(x= difflog, y = presvote)) + 
			geom_point() + 
			geom_smooth(method='lm')
		
		\includegraphics[width=1.1\textwidth]{../../../../my_answers/PS3-2.png}
		
		\item Save the residuals of the model in a separate object.	\vspace{5cm}
		PresVoteResid <- PresVoteReg\$residuals

		
		\item Write the prediction equation.
		
		y = Beta0 + Beta1*x
		presvote = 0.50758 + 0.02384*difflog
	\end{enumerate}
	
	\newpage	
\section*{Question 3}

\noindent We are interested in knowing how the vote share of the presidential candidate of the incumbent's party is associated with the incumbent's electoral success.
	\vspace{.25cm}
	\begin{enumerate}
		\item Run a regression where the outcome variable is \texttt{voteshare} and the explanatory variable is \texttt{presvote}.
		VotePresReg <- lm(voteshare~presvote, inc.sub)
		summary(VotePresReg)
		Coefficients:
							Estimate 	Std. Error	 t value Pr(>|t|)    
		(Intercept) 0.441330   0.007599   58.08   <2e-16 ***
		presvote    0.388018   0.013493   28.76   <2e-16 ***
		
		
		
			\vspace{5cm}
		\item Make a scatterplot of the two variables and add the regression line. 

		ggplot(inc.sub, aes(x= presvote, y = voteshare)) + 
			geom_point() + 
			geom_smooth(method='lm')
		
		\includegraphics[width=1.1\textwidth]{../../../../my_answers/PS3-3.png}
		
			\vspace{5cm}
		\item Write the prediction equation.
		
				y = Beta0 + Beta1*x
				voteshare =  0.4413 + presvote*0.3880
	\end{enumerate}
	

\newpage	
\section*{Question 4}
\noindent The residuals from part (a) tell us how much of the variation in \texttt{voteshare} is $not$ explained by the difference in spending between incumbent and challenger. The residuals in part (b) tell us how much of the variation in \texttt{presvote} is $not$ explained by the difference in spending between incumbent and challenger in the district.
	\begin{enumerate}
		\item Run a regression where the outcome variable is the residuals from Question 1 and the explanatory variable is the residuals from Question 2.	\vspace{6cm}
		summarylm((VoteDifResid~PresVoteResid))
		
		Residuals:
		Min       1Q   Median       3Q      Max 
		-0.25928 -0.04737 -0.00121  0.04618  0.33126 
		Coefficients:
								Estimate 	Std. Error 		t value Pr(>|t|)    
		(Intercept)   -5.934e-18  1.299e-03    0.00        1    
		PresVoteResid  2.569e-01  1.176e-02   21.84   <2e-16 ***
		
		Running on its own rather than displaying, since there is no need for it later.
		
		\item Make a scatterplot of the two residuals and add the regression line. 	\vspace{6cm}
		ggplot(inc.sub, aes(x= PresVoteResid, y = VoteDifResid)) + 
			geom_point() + 
			geom_smooth(method='lm') 
		\includegraphics[width=1.1\textwidth]{../../../../my_answers/PS3-4.png}
		\item Write the prediction equation.
		y = Beta0 + Beta1*x
		residual1 = -5.934e-18 +2.569e-01*residual2
	\end{enumerate}
	
	\newpage	

\section*{Question 5}
\noindent What if the incumbent's vote share is affected by both the president's popularity and the difference in spending between incumbent and challenger? 
	\begin{enumerate}
		\item Run a regression where the outcome variable is the incumbent's \texttt{voteshare} and the explanatory variables are \texttt{difflog} and \texttt{presvote}.	\vspace{5cm}
		summary(lm(voteshare~difflog+presvote,inc.sub))
		
		Residuals:
		Min       1Q   Median       3Q      Max 
		-0.25928 -0.04737 -0.00121  0.04618  0.33126 
		Coefficients:
							Estimate 		Std. Error 	t value Pr(>|t|)    
		(Intercept) 0.4486442  0.0063297   70.88   <2e-16 ***
		difflog     0.0355431  0.0009455   37.59   <2e-16 ***
		presvote    0.2568770  0.0117637   21.84   <2e-16 ***
		
		\item Write the prediction equation.	\vspace{5cm}
		y = beta0 + beta1*x1 + beta2*x2
		
		voteshare = 0.44864 + 0.03554*difflog + 0.25688*presvote
		
		\item What is it in this output that is identical to the output in Question 4? Why do you think this is the case?
		
		The residuals are identical in both Question 4 and Question 5. This may be true because Question 4 was created using Residuals from two problems that both used some combination of voteshare, difflog, and presvote, while Question 5 uses all three of those variables.
		
	\end{enumerate}




\end{document}
